\section{Control Cinematico del robot}
\begin{itemize}
	\item HABLAR SOBRE QUE ES EL CONTROL CINEMATICO Y LAS MOVIDAS DE LOS GENERADORES DE TRAYECTORIAS
\\Una vez estudiado el análisis cinemático del modelo de brazo manipulador, y obtenidas las ecuaciones cinemáticas inversa y directa de este, se terminará de abordar el problema cinemático, al ser capaces de desarrollar el control sobre esta; para ello, se requiere poder generar trayectorias dentro del espacio articular, con tal de que el brazo manipulador pueda cumplir una ordenes de movimiento conocidas, en este caso dadas en un espacio cartesiano.\\

\item TIPOS DE TRAYECTORIAS. COMO LA GENERAMOS
\\Visto esto, tenemos que conocer las consignas de movimiento, mejor vistas como condiciones de contorno o exigencias, que se le piden al robot; ejemplificado en este proyecto, el movimiento del efector final entre dos puntos del espacio cartesiano del robot, dado un tiempo limite para su realización.\\

	\item INTERPOLADORES DE TRAYECTORIAS
	\item IMPLEMENTACION, GRAFICAS Y CONCLUSCIONES 
\end{itemize}
	\subsection{Generador de trayectorias punto a punto}
	 
	\subsection{Generador de trayectorias lineal}
	\subsection{Pruebas y conclusiones}
		Necesidad para el trabajo-->
		 Ser capaces de que nuestro modelo de brazo manipulador ejecute ordenes de movimientos basados en trayectorias.
		  
	    Como lo conseguimos-->
	   		Generar trayectorias de referencia en el espacio articular de nuestro modelo brazo manipulador, para conseguir