\section{Anexos}
	\subsection{Codigos de programacion}
		\subsubsection{Implementación de controladores}
La implementación de los controladores en tiempo discreto se llevo a cabo a partir de una S-Function en Matlab, es decir, un proyecto en lenguaje C compilado que simule el comportamiento de lo que sería una computadora, lo que podría ser similar a un controlador real.\\
Ésta funcion fue dada por el profesor \textit{Ortega Linares, Manuel Gil} en el desarrollo de una asignatura de un año anterior y, tras analizar cuál era la mejor opcion de introducir controladores discretos para el robot, se optó por ésta.\\
Por tanto, los controladores PD y PID implementados, se muestran a continuación, los cuales serán programados en funciones que serán llamadas desde un programa principal llamado \textit{Computadora.c}, la cuál no se mostrará en éste anexo debido a que no se ha considerado importante, lo que sí se mostrará será la implementación de los controladores, los cuales se realizaron de forma absoluta y la deficion de las variables de los mismos.\\
\begin{lstlisting}[language=C,style=CStyle, caption={Declaraciones.c}]
double tiempo; /* Para tiempo de simulacion */
static double ek1[3]; /* Valores anteriores del error */

/* Valores para el PD/PID absoluto */
static double Int_e[3],dt_e[3];

/* Valores de los parametros del PD/PID */
double Ti[3],Td[3],Kp[3];

/* SE DEBERA DAR VALORES A LOS PARAMETROS EN FUNCION DEL CONTROLADOR A IMPLEMENTAR.
 * LOS VALORES DE LOS CONTROLADORES SE ENCUENTRAN EN OTRA FUNCION, SERA COPIAR-PEGAR
 */
/* Parametros del PD/PID */
Ti[0]= ;  Td[0]= ;  Kp[0]= ;
Ti[1]= ;  Td[1]= ;  Kp[1]= ;
Ti[2]= ;  Td[2]= ;  Kp[2]= ;
\end{lstlisting}

\begin{lstlisting}[language=C,style=CStyle, caption={Controlador.c}]
/* La entrada del sistema sera un vector columna de 3 componentes. */

/* Definicion de los terminos absolutos del PID */
int i;
/*  -> IMPORTANTE <-
 * SE DEBERA COMENTAR UNO DE LOS DOS CONTROLADORES CUANDO SE DESEE IMPLEMENTAR
 */

for (i=0;i<3;i++){
	Int_e[i]+=ek[i]*Tm;
	dt_e[i]=(ek[i]-ek1[i])/Tm;
  }

/* Calculo de la senal de control del PID */
for (i=0;i<3;i++){
	Ireal[i]=Kp[i]*( ek[i] + (Td[i]*dt_e[i]) + (1/Ti[i])*Int_e[i] );
  }

/* Calculo de la senal de control del PD */
for (i=0;i<3;i++){
	Ireal[i]=Kp[i]*( ek[i] + (Td[i]*dt_e[i]));
  }

/* Actualizacion de las variables anteriores */
for (i=0;i<3;i++){
	ek1[i]=ek[i];
	}
\end{lstlisting}

\begin{lstlisting}[language=C,style=CStyle, caption={Inicializaciones.c}]
int i;
for (i=0;i<3;i++){
	ek1[i]=0.;/* Valores anteriores del error */
	dt_e[i]=0.;
	Int_e[i]=0.;
}
\end{lstlisting}

\subsection{Generadores de trayectorias}
\subsection{Estimacion de parámetros}
