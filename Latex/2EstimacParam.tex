\section{Obtencion de los parametros dinamicos del robot}
Debido a que no se conocen los parametros dinamicos del robot, sera posible estimarlos mediante una serie de experimentos y aproximaciones. Se ha optado por aplicar el algoritmo de Newton-Euler, el cual es un procedimiento recursivo que se basa en la segunda ley de Newton. Este algoritmo nos dara los esfuerzos en las articulaciones.\\
Por tanto, trabajando con Matlab, si se definen todas esas variables del robot de manera simbolica y se corre el algoritmo, se obtendran los pares que sufren las articulaciones.\\
Las variables que se busca estimar del robot seran las siguientes:
\begin{center}
		$ I_{11} =
	\begin{bmatrix}
	I_{11xx} & I_{11xy} & I_{11xz}\\ 
	I_{11yx} & I_{11yy} & I_{11zz}\\ 
	I_{11zx} & I_{11zy} & I_{11zz}
	\end{bmatrix}$

	$ I_{22} =
	\begin{bmatrix}
	I_{22xx} & I_{22xy} & I_{22xz}\\ 
	I_{22yx} & I_{22yy} & I_{22zz}\\ 
	I_{22zx} & I_{22zy} & I_{22zz}
	\end{bmatrix}$
TERMINAR DE METER MATRICES LOCO
\end{center}
\begin{itemize}
	\item HABLAR UN POCO DE LA NECESIDAD DE ESTIMAR LOS PARAMETROS DE NEWTON EULER
	\item HABLAR DE COMO SE OBTUVO GAMMA SIM Y TETHA SIM
	\item HABLAR DE LA SIMPLIFICACION A PARAMETROS LI
	\item HABLAR DE LOS EXPERIMENTOS DE LOS SENOS
	\item HABLAR DE LA OPTIMIZACION Y ESTIMACION DE LOS PARAMETROS
	\item CONTRUCCION DEL MODELO A PARTIR DE LOS PARAMETROS
	\item COMPARATIVA ROBOT REAL Y MODELOS. CONCLUSIONES
\end{itemize}
	\subsection{Obtencion del modelo para estimar los parametros}
	\subsection{Estimacion de parametros dinamicos}
	\subsection{Calculos estadisticos}
	\subsection{Analisis de resultados}