\section{Obtencion de los parametros dinamicos del robot}
Debido a que no se conocen los parametros dinamicos del robot, sera posible estimarlos mediante una serie de experimentos y aproximaciones. Se ha optado por aplicar el algoritmo de Newton-Euler, el cual es un procedimiento recursivo que se basa en la segunda ley de Newton. Este algoritmo nos dara los esfuerzos en las articulaciones.\\
Por tanto, trabajando con Matlab, si se definen todas esas variables del robot de manera simbolica y se corre el algoritmo, se obtendran los pares que sufren las articulaciones.\\
Las variables que se busca estimar del robot seran las siguientes:
\begin{center}
		$ I_{11} =
	\begin{bmatrix}
	I_{11xx} & I_{11xy} & I_{11xz}\\
	I_{11yx} & I_{11yy} & I_{11zz}\\
	I_{11zx} & I_{11zy} & I_{11zz}
	\end{bmatrix} $

	$ I_{22} =
	\begin{bmatrix}
	I_{22xx} & I_{22xy} & I_{22xz}\\
	I_{22yx} & I_{22yy} & I_{22zz}\\
	I_{22zx} & I_{22zy} & I_{22zz}
	\end{bmatrix}$
TERMINAR DE METER MATRICES LOCO
\end{center}
\begin{itemize}
	\item HABLAR UN POCO DE LA NECESIDAD DE ESTIMAR LOS PARAMETROS DE NEWTON EULER
	\item HABLAR DE COMO SE OBTUVO GAMMA SIM Y TETHA SIM
	\item HABLAR DE LA SIMPLIFICACION A PARAMETROS LI
	\item HABLAR DE LOS EXPERIMENTOS DE LOS SENOS
	\item HABLAR DE LA OPTIMIZACION Y ESTIMACION DE LOS PARAMETROS
	\item CONTRUCCION DEL MODELO A PARTIR DE LOS PARAMETROS
	\item COMPARATIVA ROBOT REAL Y MODELOS. CONCLUSIONES
\end{itemize}
	\subsection{Estimacion de parametros dinamicos}
	\subsection{Cálculos estadisticos}
	\subsection{Análisis de resultados}
	A continuación, se mostrarán los parámetros obtenidos para cada configuración del robot, así cómo la covarianza con la que se han obtenido los mismos.\\
	Para evitar repetir siempre los parámetros, se irán definiendo componente a componente, es decir, a continuación se definirá tetha con todos los parámetros y se dirá en cada caso concreto la posición del parámetro obtenido en el vector, el valor de dicho parámetro y la covarianza del mismo. \\

	Además de ello, se obtendrán las ecuaciones dinámicas que definen el comportamiento dinámico de las articulaciones del robot. Para obtener éstas ecuaciones, será necesario multiplicar la matriz $\gamma$ que posteriormente se definirá por la matriz $\theta$ con valores numericos. \\
	Una vez se tengan las ecuaciones dinámicas del robot, al igual que se hizo cuando se aplicó el algoritmo de \textit{Newton-Euler}, se irán derivando las expresiones respecto las posiciones, velocidades y aceleraciones articulares para obtener las matrices de inercias, matriz de terminos de Coirolis y la matriz formada por los términos gravitarios.
\begin{equation}
\theta=
	\begin{pmatrix}
		  m_{1}s_{11z}^{2} + m_{2}s_{22x}^{2} + m_{3}s_{33x}^{2} + I_{11yy} + I_{22yy} + I{33yy} + R_{1}^{2}Jm_1 - m_2 - 1.64m_3 \\
		  Bm_{1}  \\
		  -m_{2}s_{22x}^{2} + I_{22xx} - I_{22yy} + m_{2} + m_{3} \\
		  m_{2}s_{22x}^{2} + I_{22zz} + R_{2}^{2}Jm_{2} - m_{2} - m_{3}  \\
		  Bm_{2} \\
		  - m_{3}s_{33x}^{2} + I_{33xx} - I_{33yy} + 0.64m_{3} \\
		  m_{3}s_{33x}^{2} + I_{33zz} - 0.64m_{3}  \\
		  Jm_{3}  \\
		  Bm_{3}  \\
		  -m_{2} -m_{3} + m_{2}s_{22x}  \\
		  m_{3}s_{33x} - 0.8m_{3}
	\end{pmatrix}
\end{equation}

La matriz $\gamma$ obtenida que, multiplicada por los valores numéricos de $\theta$, resultarán las ecuaciones dinámicas del robot estudiado, se muestra a continuación:

\begin{landscape}
	    \vspace*{\fill}
\resizebox{\linewidth}{!}{%
	$\gamma=
	\begin{pmatrix}
		\ddot{q_1} & 2500\dot{q_1} & \frac{\ddot{q_1}}{2} - \frac{\ddot{q_1}cos(2q_2)}{2} + \dot{q_1}\dot{q_2}sin(2q_2) & 0 & 0 & \frac{\ddot{q_1}}{2} - \frac{\ddot{q_1}cos(2q_2 + 2q_3)}{2} +\dot{q_1}\dot{q_2}sin(2q_2 + 2q_3) + \dot{q_1}\dot{q_3}sin(2q_2 + 2q_3) & 0 & 0 & 0 & -L_{2}(\ddot{q_1} + \ddot{q_1}cos(2q_2) - 2\dot{q_1}\dot{q_2}sin(2q_2)) & 2L_3\dot{q_1}\dot{q_2}sin(2q_2 + 2q_3) - L2\ddot{q_1}cos(2q_2 + q_3) - L_{3}\ddot{q_1}cos(2q_2 + 2q_3) - L2\ddot{q_1}cos(q_3) - L_{3}\ddot{q_1} + 2L_{3}\dot{q_1}\dot{q_3}sin(2q_2 + 2q_3) + L_{2}\dot{q_1}\dot{q_3}sin(q_3) + 2L_{2}\dot{q_1}\dot{q_2}sin(2q_2 + q_3) + L_{2}\dot{q_1}\dot{q_3}sin(2q_2 + q_3) \\

    0 & 0 & -\frac{\dot{q_1}^{2}sin(2q_2)}{2} & \ddot{q_2} & 900\dot{q_2} & -\frac{\dot{q_1}^{2}sin(2q_2 + 2q_3)}{2} & \ddot{q_2} + \ddot{q_3} & 0 & 0 & - L_{2}sin(2q_2)\dot{q_1}^{2} - 2L_{2}\ddot{q_2} - g*cos(q_2) & L_{2}{q_3}^{2}sin(q_3) - 2*L3*\ddot{q_3} - g*cos(q_2 + q_3) - 2L_{3}\ddot{q_2} - L_{2}\dot{q_1}^{2}*sin(2q_2 + q_3) - L_{3}\dot{q_1}^{2}sin(2q_2 + 2q_3) - 2L_2\ddot{q_2}cos(q_3) - L_{2}\ddot{q_3}cos(q_3) + 2L_{2}\dot{q_2}\dot{q_3}sin(q_3) \\

    0 & 0 & 0 & 0 & 0 & -\frac{\dot{q_1}^{2}sin(2q_2 + 2q_3)}{2} & \ddot{q_2} + \ddot{q_3} & 225*\ddot{q_3} & 225*\dot{q_3} & 0 & - 2L_{3}\ddot{q_2} - 2L_3\ddot{q_3} - g*cos(q_2 + q_3) - \frac{L_{2}\dot{q_1}^{2}sin(q_3)}{2} - L_2\dot{q_2}^{2}sin(q_3) - \frac{L_2\dot{q_1}^{2}sin(2q_2 + q_3)}{2} - L_{3}\dot{q_1}^{2}sin(2q_2 + 2q_3) - L_2\ddot{q_2}cos(q_3)\\
	\end{pmatrix}$
}
    \vspace*{\fill}
\end{landscape}

	\subsubsection{Robot Ideal con reductoras}
	Los parametros estimados, es decir, la matriz tetha linealmente independiente de 11 parámetros, obtenida para desarrollar el modelo del robot con medidas ideales con reductoras en los motores será:
	\begin{center}
		\begin{tabular}{| c | c | c |}

			\hline
			Parametro estimado & Valor obtenido & Covarianza obtenida \\
			\hline
			$\theta(1) $ & 15.6322 & 0.0338 \\
			\hline
			$\theta(2) $ & 0.0012 & 0.0218 \\
			\hline
			$\theta(3) $ & 7.389 & 0.0481 \\
			\hline
			$\theta(4) $ & 55.1139 & 0.00081 \\
			\hline
			$\theta(5) $ & 0.00085 & 0.02744 \\
			\hline
			$\theta(6) $ & 2.0841 & 0.047868 \\
			\hline
			$\theta(7) $ & -2.0414 & 0.00623 \\
			\hline
			$\theta(8) $ & 0.051 & 0.00113 \\
			\hline
			$\theta(9) $ & 0.0015 & 0.033 \\
			\hline
			$\theta(10) $ & -6.665 & 0.00054 \\
			\hline
			$\theta(11) $ & -2.222 & 0.00113 \\
			\hline


		\end{tabular}
	\end{center}
Por lo tanto, tras multiplicar $\gamma$ y $\theta$ y derivar respecto a las variables articulares, se definirán las ecuaciones dinámicas del robot ideal con reductoras cómo:\\


\resizebox{\linewidth}{!}{$%
	\begin{pmatrix}
	Kt_{1}R_{1}^{2}Im_{1} \\
	Kt_{2}R_{2}^{2}Im_{2} \\
	Kt_{3}R_{3}^{2}Im_{3}
	\end{pmatrix} =
	\begin{pmatrix}
	0.0889cos(2q_{2} + q_{3}) + 0.119*cos(2q_{2}) + 0.0889cos(q_{3}) + 0.0294cos(2q_{2} + 2q_{3}) + 1.15 & 0 & 0 \\
	0 & 0.3703cos(q_{3}) + 5.83 & 0.1851cos(q_{3}) + 0.126  \\
	0 & 0.4232cos(q_{3}) + 0.288 & 2.463
	\end{pmatrix}
	\begin{pmatrix}
	\ddot{q_{1}} \\
	\ddot{q_{2}}  \\
	\ddot{q_{3}}
	\end{pmatrix} +
	\begin{pmatrix}
	0.178\dot{q_1}\dot{q_2}sin(2q_2 + q_3) + 0.0887\dot{q_1}\dot{q_3}sin(2q_2 + q_3) + 0.235\dot{q_1}\dot{q_2}sin(2q2) + 0.0887\dot{q_3}\dot{q_1}sin(q_3) + 0.059\dot{q_1}\dot{q_2}sin(2q_2 + 2q_3) + 0.059\dot{q_1}\dot{q_3}sin(2q_2 + 2q_3) - 0.12\dot{q_1}\\
	0.0638\dot{q_2} - 0.185\dot{q_3}^{2}sin(q_3) + 0.0613\dot{q_1}^{2}sin(2q_2 + 2q_3) + 0.185\dot{q_1}^{2}sin(2q_2 + q_3) + 0.248\dot{q_1}^{2}sin(2q_2) - 0.37\dot{q_2}\dot{q_3}sin(q_3)  \\
0.0643\dot{q_3} + 0.212\dot{q_1}^{2}sin(q3) + 0.423\dot{q_2}^{2}sin(q_3) + 0.14\dot{q_1}^{2}sin(2q_2 + 2q_3) + 0.212\dot{q_1}^{2}sin(2q_2 + q_3)
	\end{pmatrix}
	\begin{pmatrix}
		\dot{q_{1}} \\
		\dot{q_{2}}  \\
		\dot{q_{3}}
	\end{pmatrix} +
\begin{pmatrix}
 		0 \\
  1.81cos(q_{2} + q_{3}) + 5.44cos(q_{2}) \\
                 4.15cos(q_{2} + q_{3})
\end{pmatrix}
$}


\newpage
\subsubsection{Robot Ideal sin reductoras}
Los parametros estimados, es decir, la matriz tetha linealmente independiente de 11 parámetros, obtenida para desarrollar el modelo del robot con medidas ideales sin reductoras en los motores será:
\begin{center}
	\begin{tabular}{| c | c | c |}

		\hline
		Parametro estimado & Valor obtenido & Covarianza obtenida \\
		\hline
		$\theta(1) $ & -9.31476 & 0.00364 \\
		\hline
		$\theta(2) $ & 0.001193 & 2.868 \\
		\hline
		$\theta(3) $ & 7.3803 & 0.00036 \\
		\hline
		$\theta(4) $ & -7.2341 & 0.00369 \\
		\hline
		$\theta(5) $ & 0.00121 & 5.890 \\
		\hline
		$\theta(6) $ & 2.078 & 0.00358 \\
		\hline
		$\theta(7) $ & -2.0335 & 0.00359 \\
		\hline
		$\theta(8) $ & 0.051 & 0.0148 \\
		\hline
		$\theta(9) $ & 0.00146 & 2.19 \\
		\hline
		$\theta(10) $ & -6.6585 & 0.003621 \\
		\hline
		$\theta(11) $ & -2.222 & 0.00356 \\
		\hline
	\end{tabular}
\end{center}
Por lo tanto, tras multiplicar $\gamma$ y $\theta$ y derivar respecto a las variables articulares, se definirán las ecuaciones dinámicas del robot ideal con reductoras cómo:\\

\resizebox{\linewidth}{!}{$%
	\begin{pmatrix}
	Kt_{1}R_{1}^{2}Im_{1} \\
	Kt_{2}R_{2}^{2}Im_{2} \\
	Kt_{3}R_{3}^{2}Im_{3}
	\end{pmatrix} =
	\begin{pmatrix}
	4.434cos(2q_{2} + q_{3}) + 5.937cos(2q_{2}) + 4.434cos(q_{3}) + 1.469cos(2q_{2} + 2q_{3}) + 7.693 &  0 & 0 \\
		0 & 11.08cos(q_{3}) + 18.99 & 5.542cos(q_{3}) + 3.784 \\
		0 & 6.334cos(q3) + 4.325 &   4.47
	\end{pmatrix}
	\begin{pmatrix}
	\ddot{q_{1}} \\
	\ddot{q_{2}}  \\
	\ddot{q_{3}}
	\end{pmatrix} +
	\begin{pmatrix}
-8.84\dot{q_1}\dot{q_2}sin(2q_2 + q_3) - 4.429\dot{q_1}\dot{q_3}sin(2q_2 + q_3)  -11.85\dot{q_1}\dot{q_2}sin(2q_2) - 4.429\dot{q_1}\dot{q_3}sin(q_3) -2.924\dot{q_1}\dot{q_2}sin(2q_2 + 2q_3) - 2.924\dot{q_1}\dot{q_3}sin(2q_2 + 2q_3) 0.002387\dot{q_1} \\

0.00304\dot{q_2} - 5.54\dot{q_3}^{2}sin(q_{3}) + 1.84\dot{q_{1}}^{2}sin(2q_{2} + 2q_{3}) + 5.54\dot{q_1}^{2}sin(2q_{2} + q_{3}) + 7.42\dot{q_1}^{2}sin(2q_{2}) - 11.1\dot{q_{2}}\dot{q_{3}}sin(q_{3}) \\

0.00418\dot{q_3} + 3.17\dot{q_1}^{2}sin(q_{3}) + 6.33\dot{q_2}^{2}sin(q_{3}) + 2.1\dot{q_1}^{2}sin(2q_{2} + 2q_{3}) + 3.17\dot{q_1}^{2}sin(2q_{2} + q_{3})
	\end{pmatrix}
	\begin{pmatrix}
		\dot{q_{1}} \\
		\dot{q_{2}}  \\
		\dot{q_{3}}
	\end{pmatrix} +
\begin{pmatrix}
	0 \\
54.3cos(q_{2} + q_{3}) + 163cos(q_{2}) \\
62.1cos(q_{2} + q_{3})
\end{pmatrix}
$}


\newpage
\subsubsection{Robot Real con reductoras}
Los parametros estimados, es decir, la matriz tetha linealmente independiente de 11 parámetros, obtenida para desarrollar el modelo del robot con medidas reales con reductoras en los motores será:
\begin{center}
	\begin{tabular}{| c | c | c |}

		\hline
		Parametro estimado & Valor obtenido & Covarianza obtenida \\
		\hline
		$\theta(1) $ & 16.995 & 4.0573 \\
		\hline
		$\theta(2) $ & 0.00122 & 0.2538 \\
		\hline
		$\theta(3) $ & 12.393 & 1.291 \\
		\hline
		$\theta(4) $ & 38.28 & 1.9472 \\
		\hline
		$\theta(5) $ & 0.00129 & 0.941 \\
		\hline
		$\theta(6) $ & 1.434 & 0.917 \\
		\hline
		$\theta(7) $ & 4.0372 & 4.545 \\
		\hline
		$\theta(8) $ & 0.0491 & 1.234 \\
		\hline
		$\theta(9) $ & 0.00151 & 1.468 \\
		\hline
		$\theta(10) $ & -6.6722 & 0.003858 \\
		\hline
		$\theta(11) $ & -2.199 & 0.008916 \\
		\hline
	\end{tabular}
\end{center}
Por lo tanto, tras multiplicar $\gamma$ y $\theta$ y derivar respecto a las variables articulares, se definirán las ecuaciones dinámicas del robot ideal con reductoras cómo:\\

\resizebox{\linewidth}{!}{$%
	\begin{pmatrix}
	Kt_{1}R_{1}^{2}Im_{1} \\
	Kt_{2}R_{2}^{2}Im_{2} \\
	Kt_{3}R_{3}^{2}Im_{3}
	\end{pmatrix} =
	\begin{pmatrix}
		0.088cos(2q_2 + q_3) + 0.019cos(2q_2) + 0.088cos(q_3) + 0.0417cos(2q_{2} + 2q_{3}) + 1.29 &      0 &       0 \\
	   0   &    0.366cos(q_3) + 4.6    &     0.183cos(q_3) + 0.293 \\
		0    &    0.419cos(q_3) + 0.67   &           2.78
	\end{pmatrix}
	\begin{pmatrix}
	\ddot{q_{1}} \\
	\ddot{q_{2}}  \\
	\ddot{q_{3}}
	\end{pmatrix} +
	\begin{pmatrix}
-0.1758\dot{q_1}\dot{q_2}sin(2q_2 + q_3) - 0.08791\dot{q_1}\dot{q_3}sin(2q_2 + q_3) - 0.03796\dot{q_1}\dot{q_2}sin(2q_2) - 0.08791\dot{q_1}\dot{q_3}sin(q_3) -0.08325\dot{q_1}\dot{q_2}sin(2q_2 + 2q_3) - 0.08325\dot{q_1}\dot{q_3}sin(2q_2 + 2q_3) + 0.1223*qd1)\\

0.0969\dot{q_2} - 0.183\dot{q_3}^{2}sin(q_3) + 0.0868\dot{q_1}^{2}sin(2q_{2} + 2q_{3}) + 0.183\dot{q_1}^{2}sin(2q_2 + q_3) + 0.0396\dot{q_1}^{2}sin(2q_{2}) - 0.366\dot{q_2}\dot{q_3}sin(q_3)\\

0.0648\dot{q_3} + 0.209\dot{q_1}^{2}sin(q_3) + 0.419\dot{q_2}^{2}sin(q_3) + 0.199\dot{q_1}^{2}sin(2q_2 + 2q_3) + 0.209\dot{q_1}^{2}sin(2q_2 + q_3)
	\end{pmatrix}
	\begin{pmatrix}
		\dot{q_{1}} \\
		\dot{q_{2}}  \\
		\dot{q_{3}}
	\end{pmatrix} +
\begin{pmatrix}
	0																	\\
1.796cos(q_2 + q_3) + 5.449cos(q_2) \\
4.105cos(q_2 + q_3)
\end{pmatrix}
$}



\newpage
\subsubsection{Robot Real sin reductoras}
Los parametros estimados, es decir, la matriz tetha linealmente independiente de 11 parámetros, obtenida para desarrollar el modelo del robot con medidas reales sin reductoras en los motores será: 
\begin{center}
	\begin{tabular}{| c | c | c |}

		\hline
		Parametro estimado & Valor obtenido & Covarianza obtenida \\
		\hline
		$\theta(1) $ & - & - \\
		\hline
		$\theta(2) $ & - & - \\
		\hline
		$\theta(3) $ & - & - \\
		\hline
		$\theta(4) $ & - & - \\
		\hline
		$\theta(5) $ & - & - \\
		\hline
		$\theta(6) $ & - & - \\
		\hline
		$\theta(7) $ & - & - \\
		\hline
		$\theta(8) $ & - & - \\
		\hline
		$\theta(9) $ & - & - \\
		\hline
		$\theta(10) $ & - & - \\
		\hline
		$\theta(11) $ & - & - \\
		\hline
	\end{tabular}
\end{center}
Por lo tanto, tras multiplicar $\gamma$ y $\theta$ y derivar respecto a las variables articulares, se definirán las ecuaciones dinámicas del robot ideal con reductoras cómo:\\

\resizebox{\linewidth}{!}{$%
	\begin{pmatrix}
	Kt_{1}R_{1}^{2}Im_{1} \\
	Kt_{2}R_{2}^{2}Im_{2} \\
	Kt_{3}R_{3}^{2}Im_{3}
	\end{pmatrix} =
	\begin{pmatrix}
	tra & tra & malamente \\
	tra & tra & malamente  \\
	tra & tra & malamente
	\end{pmatrix}
	\begin{pmatrix}
	\ddot{q_{1}} \\
	\ddot{q_{2}}  \\
	\ddot{q_{3}}
	\end{pmatrix} +
	\begin{pmatrix}
	tra  \\
	tra   \\
	tra
	\end{pmatrix}
	\begin{pmatrix}
		\dot{q_{1}} \\
		\dot{q_{2}}  \\
		\dot{q_{3}}
	\end{pmatrix} +
\begin{pmatrix}
tra  \\
tra   \\
tra
\end{pmatrix}
$}



\newpage
\subsection{Verificación de los modelos obtenidos}
