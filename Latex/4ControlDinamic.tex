\section{Control Dinámico del brazo}
En esta última parte del proyecto, una vez se conoce el modelo del robot en las diferentes configuraciones, se pasará a buscar implementar un control dinámico sobre el mismo.Para poder implementar controladores sobre nuestro robot, será necesario obtener una función de transferencia matemática a modo de modelo que se asemeje al robot real.\\
Una vez se tenga un modelo lineal de cada articulación del robot, junto con el generador de trayectorias creado anteriormente, se buscará que el robot siga una trayectoria predefinida.\\

	\subsection{Obtención del modelo lineal de las articulaciones del brazo}
Para obtener la función de transferencia de cada articulación del robot, se linealizará la ecuación dinamica que define cada motor en un punto de equilibrio en torno a velocidades nulas. Por lo tanto, las consideraciones que se tendrán en cuenta para linealizar la ecuación dinamica que define el comportamiento de cada articulacion del robot son:\\
\begin{itemize}
	\item Velocidades de equilibrio
	\begin{center}
		$ \dot{q_{eq}}=0 rad/s $\\
		$ \dot{q} =\dot{q_{eq}}+\Delta\dot{q}$
	\end{center}

	\item Aceleraciones de equilibrio
\begin{center}
	$ \ddot{q_{eq}}=0 rad/s $\\
	$  $
	$ \ddot{q} =\ddot{q_{eq}}+\Delta\ddot{q}$
\end{center}

\end{itemize}

Ademas de ello, se aplicarán una serie de simplificacines a la ecuación dinamica. A continuación, se mostrarán las ecuaciones dimámicas de los motores:\\
\begin{center}
	
	\begin{pmatriz}
	$ \tau1$ \\
	$ \tau2$ \\
	$ \tau3$ 
	\end{pmatriz}
	=
	\begin{pmatriz}
		$ Kt_{1}R_{1}Im_{1} $ \\
		$ Kt_{2}R_{2}Im_{2} $ \\
		$ Kt_{3}R_{3}Im_{3} $ 
	\end{pmatriz}
	=
	 \begin{pmatriz}
	 	$ Ma$
	 \end{pmatriz}
\end{center}
	\subsection{Diseño de controladores}
	\subsection{FANTASIA VARIADA DE GIL}